% !Mode:: "TeX:UTF-8"
% !TEX root = ../main.tex

\section{Test Unicode}

八思巴字(A840-A877):

{\psp ꡀꡁꡂꡃꡄꡅꡆꡇꡈꡉꡊꡋꡌꡍꡎꡏꡐꡑꡒꡓꡔꡕꡖꡗꡘ

ꡙꡚꡛꡜꡝꡞꡟꡠꡡꡢꡣꡤꡥꡦꡧꡨꡩꡪꡫꡬꡭꡮꡯꡰꡱꡲꡳ꡴꡵꡶꡷}


注音符号(3100-312F):\*ㄅㄆㄇㄈㄉㄊㄋㄌㄍㄎㄏㄐㄑㄒㄓㄔㄕㄖㄗㄘㄙㄚㄛㄜㄝㄞㄟㄠㄡㄢㄣㄤㄥㄦㄧㄨㄩㄪㄫㄬㄭㄮㄯ

注音扩展(31A0-31BF):ㆠㆡㆢㆣㆤㆥㆦㆧㆨㆩㆪㆫㆬㆭㆮㆯㆰㆱㆲㆳㆴㆵㆶㆷㆸㆹㆺㆻㆼㆽㆾㆿ

CJK笔画(31C0-31EF):㇀㇁㇂㇃㇄㇅㇆㇇㇈㇉㇊㇋㇌㇍㇎㇏㇐㇑㇒㇓㇔㇕㇖㇗㇘㇙㇚㇛㇜㇝㇞㇟㇠㇡㇢㇣——㇯

CJK扩展A(3400-4DBF):㐀㐁㐂㐃㐄㐅㐆㐇㐈㐉㐊㐋㐌㐍㐎㐏——䶰䶱䶲䶳䶴䶵䶶䶷䶸䶹䶺䶻䶼䶽䶾䶿

CJK统一表意文字(4E00-9FFF):一丁丂七丄丅丆万丈三上下丌不与丏——鿰鿱鿲鿳鿴鿵鿶鿷鿸鿹鿺鿻鿼鿽鿾鿿

中日韩兼容表意文字(F900-FAFF):豈更車賈滑串句龜龜契金喇奈懶癩羅——𢡄𣏕㮝䀘䀹𥉉𥳐𧻓齃龎\*	

CJK扩展B(20000-2A6DF):𠀀𠀁𠀂𠀃𠀄𠀅𠀆𠀇𠀈𠀉𠀊𠀋𠀌𠀍𠀎𠀏——𪛐𪛑𪛒𪛓𪛔𪛕𪛖𪛗𪛘𪛙𪛚𪛛𪛜𪛝𪛞𪛟

CJK扩展C(2A700-2B73F):𪜀𪜁𪜂𪜃𪜄𪜅𪜆𪜇𪜈𪜉𪜊𪜋𪜌𪜍𪜎𪜏——𫜰𫜱𫜲𫜳𫜴𫜵𫜶𫜷𫜸𫜹\*

CJK扩展D(2B740-2B81F):𫝀𫝁𫝂𫝃𫝄𫝅𫝆𫝇𫝈𫝉𫝊𫝋𫝌𫝍𫝎𫝏——𫠐𫠑𫠒𫠓𫠔𫠕𫠖𫠗𫠘𫠙𫠚𫠛𫠜𫠝\*

CJK扩展E(2B820-2CEAF):𫠠𫠡𫠢𫠣𫠤𫠥𫠦𫠧𫠨𫠩𫠪𫠫𫠬𫠭𫠮𫠯——𬺐𬺑𬺒𬺓𬺔𬺕𬺖𬺗𬺘𬺙𬺚𬺛𬺜𬺝𬺞𬺟𬺠𬺡\*

CJK扩展F(2CEB0-2EBEF):𬺰𬺱𬺲𬺳𬺴𬺵𬺶𬺷𬺸𬺹𬺺𬺻𬺼𬺽𬺾𬺿——𮯐𮯑𮯒𮯓𮯔𮯕𮯖𮯗𮯘𮯙𮯚𮯛𮯜𮯝𮯞𮯟𮯠\*

CJK扩展I(2EBF0-2EE5F):𮯰𮯱𮯲𮯳𮯴𮯵𮯶𮯷𮯸𮯹𮯺𮯻𮯼𮯽𮯾𮯿——𮹐𮹑𮹒𮹓𮹔𮹕𮹖𮹗𮹘𮹙𮹚𮹛𮹜𮹝\*

CJK扩展G(30000-3134F):𰀀𰀁𰀂𰀃𰀄𰀅𰀆𰀇𰀈𰀉𰀊𰀋𰀌𰀍𰀎𰀏——𱍀𱍁𱍂𱍃𱍄𱍅𱍆𱍇𱍈𱍉𱍊\*

CJK扩展H(31350-323AF):𱍐𱍑𱍒𱍓𱍔𱍕𱍖𱍗𱍘𱍙𱍚𱍛𱍜𱍝𱍞𱍟——𲎠𲎡𲎢𲎣𲎤𲎥𲎦𲎧𲎨𲎩𲎪𲎫𲎬𲎭𲎮𲎯

谚文字母(Hangeul Jamo,1100-11FF):
{\oldjamo ᄀᄁᄂᄃᄄᄅᄆᄇᄈᄉᄊᄋᄌᄍᄎᄏ

——ᇰᇱᇲᇳᇴᇵᇶᇷᇸᇹᇺᇻᇼᇽᇾᇿ}

表意文字描述(IDC,2FF0-2FFF):⿰⿱⿲⿳⿴⿵⿶⿷⿸⿹⿺⿻⿼⿽⿾⿿

谚文兼容字母(Hangul Compatibility Jamo,3130-318F):\* 

{\oldjamo ㄱㄲㄳㄴㄵㄶㄷㄸㄹㄺㄻㄼㄽㄾㄿ——ㆀㆁㆂㆃㆄㆅㆆㆇㆈㆉㆊㆋㆌㆍㆎ }\*

谚文字母扩展A(A960-A97F):

{\oldjamo ꥠꥡꥢꥣꥤꥥꥦꥧꥨꥩꥪꥫꥬꥭꥮ
ꥰꥱꥲꥳꥴꥵꥶꥷꥸꥹꥺꥻꥼ}

谚文音节(AC00-D7AF):가각갂갃간갅갆갇갈갉갊갋갌갍갎갏——힠힡힢힣\*

谚文字母扩展B(D7B0-D7FF):

{\oldjamo ힰힱힲힳힴힵힶힷힸힹힺힻힼힽힾힿ——ퟰퟱퟲퟳퟴퟵퟶퟷퟸퟹퟺퟻ}

\subsection{Old Jamo}

旧谚文(另独立编码在私用区「汉阳PUA」E0BC-F8F7,此示例为字母拼成):

摘自\href{https://charset.fandom.com/ko/wiki/한양_PUA}{https://charset.fandom.com/ko/wiki/한양\_PUA}

{
    \oldjamo
F28 ᅎᆑ	ᅎᆑᆫ	ᅎᅳ	ᅎᅳᆫ	ᅎᅳᆺ	ᅎᅳᇫ	ᅎᅵ	ᅎᅵᆫ	ᅎᅵᆷ	ᅎᅵᇢ	ᅎᅵᇫ	ᅏᅡ	ᅏᅡᆫ	ᅏᅡᆷ	ᅏᅡᇢ	ᅏᅡᇰ

F29	ᅏᅢ	ᅏᅣ	ᅏᅣᆷ	ᅏᅣᇰ	ᅏᅥ	ᅏᅥᆫ	ᅏᅧ	ᅏᅧᆫ	ᅏᅨ	ᅏᅮ	ᅏᅱ	ᅏᅲ	ᅏᅲᇰ	ᅏᆑ	ᅏᆑᆫ	ᅏᅳ

F2A	ᅏᅳᇫ	ᅏᅳᇰ	ᅏᅵ	ᅏᅵᆫ	ᅏᆞ	ᅏᆡ	ᅐᅡ	ᅐᅡᆫ	ᅐᅡᆮ	ᅐᅡᆷ	ᅐᅡᇢ	ᅐᅡᆸ	ᅐᅢ	ᅐᅣ	ᅐᅣᆷ	ᅐᅣᇢ

F2B	ᅐᅣᇰ	ᅐᅧ	ᅐᅧᆫ	ᅐᅩ	ᅐᅪ	ᅐᅪᆼ	ᅐᅮ	ᅐᅮᆼ	ᅐᅮᇹ	ᅐᅲ	ᅐᅲᆨ	ᅐᅲᆫ	ᅐᅲᇰ	ᅐᆑ	ᅐᆑᆫ	ᅐᅳ

F2C	ᅐᅳᆷ	ᅐᅳᇫ	ᅐᅵ	ᅐᅵᆫ	ᅐᅵᇫ	ᅐᅵᇰ	ᅑᅡ	ᅑᅡᆫ	ᅑᅡᇂ	ᅑᅢ	ᅑᅣ	ᅑᅣᆷ	ᅑᅣᇢ	ᅑᅣᆸ	ᅑᅣᇰ	ᅑᅧ

F2D	ᅑᅧᆫ	ᅑᅮ	ᅑᅱ	ᅑᅲ	ᅑᅲᇰ	ᅑᆑ	ᅑᆑᆫ	ᅑᅳ	ᅑᅳᆷ	ᅑᅳᇢ	ᅑᅵ	ᅑᅵᆫ	ᅑᅵᆷ	ᅑᅵᇫ	ᅑᅵᇰ	차ᇙ

F2E	차ᇢ	차ᇦ	차ᇰ	차ᇹ	ᄎᅶ	챠ᇙ	챠ᇦ	챠ᇰ	챠ᇱ	ᄎᅸ	ᄎᅸᆯ	ᄎᆤ	처ᇫ	처ᇰ	ᄎᅼ	ᄎᅼᆫ

F2F	쳐ퟍ	쳐ퟎ	쳐ퟏ	쳐ᇙ	쳐ᇢ	쳐ᇰ	쳐ᇱ	쳐ᇹ	ᄎᅽ	ᄎᅾ	초ᇙ	초ᇢ	초ퟨ	초ᇫ	초ᇰ	촤ᇙ

F30	촤ᇰ	ᄎᆂ	쵸ᇢ	쵸ᇰ	ᄎᆇ	ᄎᆈ	ᄎᆈᆫ	추ᇙ	추ᇠ	추ᇢ	추ᇰ	추ᇹ	ᄎᆉ	ᄎᆉᆫ	ᄎᆉᆼ	ᄎᆊ

F31	춰ᇹ	ᄎힵ	ᄎᆌ	취ᇙ	취ᇹ	츄ᇙ	츄ᇢ	츄ᇰ	츄ᇹ	ᄎᆎ	ᄎᆎᆫ	ᄎᆏ	ᄎᆏᆷ	ᄎᆐ	ᄎᆑ	ᄎᆑᆫ

F32	ᄎᆑᆯ	ᄎᆑᇙ	ᄎᆑᆷ	ᄎᆒ	ᄎᆒᆼ	ᄎᆔ	ᄎᆔᆫ	ᄎᆔᆯ	ᄎᆔᆼ	츠ퟛ	츠ᇙ	츠ퟝ	츠ᇢ	츠ᇫ	츠ᇰ	ᄎᆕ
}

ᅎᆑ % 古谚文 初中终编码

십년 % 有独立码位的现代韩语音节

중국 % 现代韩语音节,初中终编码

{\oldjamo
幫바ᇰ 滂파ᇰ 並 삐ᇰ 明미ᇰ

非ᄫᅵ (敷)  奉ᄬᅳᇰ 微ᄝᅵ

端둰 透트ᇢ 定띠ᇰ 泥니

精ᅎᅵᇰ 清ᅔᅵᇰ 從ᅏᅮᇰ 心ᄼᅵᆷ 邪ᄽᅧ

照ᅐᅣᇢ 穿ᅕᆑᆫ 牀ᅑᅪᇰ 審ᄾᅵᆷ 禪ᄿᅧᆫ

見견 溪키 羣뀬 疑ᅌᅵ

影ᅙᅵᇰ 曉햐ᇢ 匣ᅘᅣᇹ 喻유

來래 日ᅀᅵᇹ
}

{\oldjamo
國 귁〮 之 징 語 ᅌᅥᆼ〯 音 ᅙᅳᆷ이〮 \par
異 잉〮乎 ᅘᅩᆼ 中 듀ᇰ 國 귁〮ᄒᆞ〮야〮}
    
\ifXeTeX
% \begin{korean}
\oldjamo
\begin{jamotext}
    na/ras;mar:ss@/mi;
    中dyuf/國guig;ei; dar/a;
    文mun/字jj@q;oa;ro; se/rv/ s@/m@s/di; a/ni;h@r/ss@i;
    i;ren jyen/c@;ro; e/rin; 百b@ig;姓syef;i;
    ni/rv/go;jye; horx; bai; i/sye;do;
    m@/c@m;nai: jei bdv;dvr; si/re; pye/di; mod:h@rx no;mi; ha/ni;ra
\end{jamotext}
% \end{korean}
\fi

\ifLuaTeX
\begin{jamotext}
        na/ras;mar:ss@/mi;
        中dyuf/國guig;ei; dar/a;
        文mun/字jj@q;oa;ro; se/rv/ s@/m@s/di; a/ni;h@r/ss@i;
        i;ren jyen/c@;ro; e/rin; 百b@ig;姓syef;i;
        ni/rv/go;jye; horx; bai; i/sye;do;
        m@/c@m;nai: jei bdv;dvr; si/re; pye/di; mod:h@rx no;mi; ha/ni;ra
\end{jamotext}
\fi